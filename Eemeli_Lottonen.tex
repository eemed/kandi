% !TEX encoding = UTF-8 Unicode
%%%%%%%%%%%%%%%%%%%%%%%%%%%%%%%%%%%%%%%%%%%%%%%%%%%%%%%%%%%%%%%%%%%%
%
% Tampereen yliopisto
% Luonnontieteiden tiedekunta
% Matematiikka
%
% Kandidaattitutkielman malli.
% Tutkielman matemaattinen sisältö on prof. Seppo Hyyrön.
%
% Päivitetty 23.9.2018
%
%%%%%%%%%%%%%%%%%%%%%%%%%%%%%%%%%%%%%%%%%%%%%%%%%%%%%%%%%%%%%%%%%%%%
% Yli 60-sivuiset pro gradu -tutkielmat painetaan kaksipuolisina. Jos pro gradu -tutkielmassa on korkeintaan 60 sivua, se painetaan yksipuolisena eli vain paperin yhdelle puolelle. Näin sen vuoksi, että kovin ohuen tutkielman selkään ei mahdu tekijän nimeä. Sivumäärään lasketaan tutkielman kaikki sivut nimiösivusta viimeiseen liitesivuun asti.
\documentclass[a4paper,12pt,leqno,oneside]{report} % jos korkeintaan 60 sivua
%\documentclass[a4paper,12pt,leqno,twoside]{report} % jos yli 60 sivua
\usepackage[utf8]{inputenc}
\usepackage[T1]{fontenc}
\usepackage[finnish]{babel}
\usepackage[intlimits]{amsmath}
%\usepackage{amssymb} % Ei tarvita, kun käytetään makropakettia newtxmath.
\usepackage{amsthm}

% Fontteina Times ja Helvetica.
\usepackage[scale=0.94]{tgheros} % Helvetica tekstifontit.
\usepackage{newtxtext} % Times tekstifontit.
\usepackage[vvarbb]{newtxmath} % Times matematiikkafontit.

% Vaihtoehtoisesti fontteina voi käyttää LaTeXin oletusarvoista Computer Modern -kirjainperhettä (poista tällöin edelliset komennot).
%\usepackage{lmodern}

\usepackage{bm} % Matematiikkatilan lihavat kursiivikirjaimet.
\usepackage{enumerate} % Luetelmanumeroiden muokkaamiseen.
%\usepackage{pdfpages} % Pdf-muotoisten liitteiden liittämiseen.

% Väliotsikoiden tyyliä muokataan makropaketilla titlesec. Jos tarvitset otsikkotasoa \subsubsection, niin muuta optio small muotoon medium.
\usepackage[rm,bf,small,pagestyles]{titlesec}
\titleformat{\chapter}{\normalfont\rmfamily\bfseries\LARGE}{\thechapter}{1em}{}
\titlespacing*{\chapter}{0pt}{-25pt}{30pt}

% Ulkoiset kuvatiedostot liitetään makropaketin graphicx komennolla \includegraphics.
\usepackage{graphicx}
%\graphicspath{{Kuvat/}} % Ulkoisten kuvatiedostojen kansio.

% Kelluvien kuvien ja taulukoiden oletusarvoista sijoittelua voi säätää makropaketin float komennolla \floatplacement.
\usepackage{float}
\floatplacement{figure}{htb}	% oletusarvo on tbp
\floatplacement{table}{htb}		% oletusarvo on tbp

% Kuvien ja taulukoiden otsikot muotoillaan makropaketilla caption.
\usepackage[%
	margin=\leftmargini,
	labelfont=bf,
	labelsep=period,
	tableposition=top,
]{caption}

% Hypertekstilinkkejä ja url-osoitteita voi tehdä makropaketilla hyperref, jota on kutsuttava viimeiseksi. Optioilla pdftitle ja pdfauthor pdf-tiedostoon lisätään tutkielman ja tekijän nimet metatietoina.
\usepackage[%
	pdftitle={Algebrallista koodausteoriaa},
	pdfauthor={Eemeli Lottonen},
	hidelinks,
	pdfpagemode=UseNone,
	pdfstartview=FitH]{hyperref}
% Verkko-osoitteet tulostetaan samalla fontilla kuin muukin teksti:
\urlstyle{same}

% Kasvatetaan palstan korkeutta neljä riviä:
\addtolength{\textheight}{4\baselineskip}
\addtolength{\topmargin}{-3.5\baselineskip}

\setlength{\textwidth}{401pt} % Palstan leveys.
\setlength{\footskip}{4\baselineskip} % Sivunumeron etäisyys palstan alareunasta.

% Asetetaan marginaalit yhtäsuuriksi:
\setlength{\oddsidemargin}{0.5\paperwidth}
\addtolength{\oddsidemargin}{-0.5\textwidth}
\addtolength{\oddsidemargin}{-1in}
\setlength{\evensidemargin}{\oddsidemargin}

% Kandidaattitutkielmassa tarvitaan harvennettua riviväliä, jotta korjausmerkinnöille jää tilaa. Gradussa riviväliä ei harvenneta, joten seuraava rivi jätetään silloin pois.
\linespread{1.35}\selectfont % vain kandidaattitutkielmassa

% Lauseiden teksti kursivoidaan:
\theoremstyle{plain}
\newtheorem{lause}{Lause}[chapter]

% Määritelmien ja esimerkkien tekstiä ei yleensä kursivoida.
\theoremstyle{definition}
\newtheorem{maaritelma}{Määritelmä}[chapter]
\newtheorem{esimerkki}{Esimerkki}[chapter]

% Huomautuksia ei numeroida:
\theoremstyle{remark}
\newtheorem*{huomautus}{Huomautus}

% Kaavat numeroidaan luvuittain:
\numberwithin{equation}{chapter}

% Pitkä yhtälöketju voidaan jakaa eri sivuille:
\allowdisplaybreaks[1]

% Muutetaan lähdeluettelon otsikko, joka on oletusarvoisesti "Kirjallisuutta".
\AtBeginDocument{\renewcommand*{\bibname}{Lähteet}}

% Komento \Kuukausi tulostaa nykyisen kuukauden nimen isolla alkukirjaimella kirjoitettuna.
\newcommand*{\Kuukausi}{\ifcase\ \month\ \or\ Tammi\or\ Helmi\or\ Maalis\or\ Huhti\or\ Touko\or\ Kesä\or\ Heinä\or\ Elo\or\ Syys\or\ Loka\or\ Marras\or\ Joulu\fi kuu}

% Joitain komentoja matemaattisia merkintöjä varten:
\newcommand*{\Nset}{\mathbb{N}}  % luonnollisten lukujen joukko
\newcommand*{\Zset}{\mathbb{Z}}  % kokonaislukujen joukko
\newcommand*{\Qset}{\mathbb{Q}}  % rationaalilukujen joukko
\newcommand*{\Rset}{\mathbb{R}}  % reaalilukujen joukko
\newcommand*{\Cset}{\mathbb{C}}  % kompleksilukujen joukko

% Makropaketin amsmath käyttöohjeissa (amsldoc.pdf s. 18--19) suositellaan itseisarvolle ja normille seuraavia komentoja:
\newcommand*{\abs}[1]{\left\lvert#1\right\rvert}   % itseisarvo
\newcommand*{\norm}[1]{\left\lVert#1\right\rVert}  % normi

% Matriiseille ja vektoreille kannattaa määritellä oma komentonsa. Komennon \bm asemesta voi tässä käyttää komentoa \mathbf, jos halutaan käyttää pystyjä eikä kursivoituja kirjaimia.
\newcommand*{\mx}{\bm}  % matriisi tai vektori

% Suomessa käytettävä integraaliinsijoitusmerkki saadaan alla olevalla komennolla \sijoitus{alaraja}{yläraja}. Tämän kanssa on hyvä käyttää amsmath-makropaketin optiota [intlimits], jolla integrointirajat asetetaan integraalimerkin ylä- ja alapuolelle eikä viereen kuten LaTeXissa oletusarvoisesti.
\makeatletter
\@ifpackageloaded{newtxmath}
{\newcommand*{\sijoitus}[2]{\mathop{\bigg/}\limits_{\mspace{-11mu}#1}^{\mspace{10mu}#2}}}
{\newcommand*{\sijoitus}[2]{\mathop{\Big/}\limits_{\mspace{-19mu}#1}^{\mspace{19mu}#2}}}
\makeatother

%%%%%%%%%%%%%%%%%%%%%%%%%%%%%%%%%%%%%%%%%%%%%%%%%%%%%%%%%%%%%%%%%%%%
\begin{document}

% Tutkielman nimiösivu:
\begin{titlepage}
\large\bfseries\centering

\hrule height 1pt
\medskip

TAMPEREEN YLIOPISTO\\
Kandidaattitutkielma
% Pro gradu -tutkielma

\medskip
\hrule height 1pt

    \vspace{\fill}

    \raisebox{1.5cm}[0pt][0pt]{Eemeli Lottonen}\\[-\baselineskip]
    {\LARGE Algebrallista koodausteoriaa}

\vspace{\fill}

\hrule height 1pt
\medskip

Luonnontieteiden tiedekunta\\
Matematiikka\\
\Kuukausi\ \the\year\

\medskip
\hrule height 1pt
\end{titlepage}

% Lasketaan nimiösivu mukaan sivunumerointiin:
\setcounter{page}{2}

\cleardoublepage\

%%%%%%%%%%%%%%%%%%%%%%%%%%%%%%%%%%%%%%%%%%%%%%%%%%%%%%%%%%%%%%%%%%%%
% Sisällysluettelo.
\tableofcontents

\cleardoublepage\


%%%%%%%%%%%%%%%%%%%%%%%%%%%%%%%%%%%%%%%%%%%%%%%%%%%%%%%%%%%%%%%%%%%%
%           https://matematiikkalehtisolmu.fi/sanakirja
%%%%%%%%%%%%%%%%%%%%%%%%%%%%%%%%%%%%%%%%%%%%%%%%%%%%%%%%%%%%%%%%%%%%
\chapter{Johdanto}

Koodausteoriassa käsitellään tiedonsiirron virhealttiutta ja erilaisia tapoja havaita ja korjata lähetyksessä tapahtuneita virheitä. Perusideana on että, lähdetin lisää viestiin ylimääräistä dataa, jolla viesti voidaan korjata. Tätä korjaus bittien lisäystä viestiin kutsutaan koodaamiseksi. Ylimääräisten bittien määrästä riippuen vastaanottimessa on mahdollista havaita, että viestissä on tapahtunut virheitä tai jopa korjaamaan tapahtuneet virheet. Jotta viesti pystytään kokonaan korjaamaan täytyy korjaus bittejä olla huomattavan paljon viestin pituuteen nähden.

Mitä tutkielmassa tehdään

Edellytykset 

%%%%%%%%%%%%%%%%%%%%%%%%%%%%%%%%%%%%%%%%%%%%%%%%%%%%%%%%%%%%%%%%%%%%


%%%%%%%%%%%%%%%%%%%%%%%%%%%%%%%%%%%%%%%%%%%%%%%%%%%%%%%%%%%%%%%%%%%%
\chapter{Lineaariset binääri koodit}\label{ch: Lineaariset binääri koodit}

\begin{esimerkki}
    Esimerkki mistä puhutaan
\end{esimerkki}

\section{Tarvittavien käsitteiden määritelmiä}
\begin{maaritelma}
    Olkoon $A$ jokin äärellinen joukko, kutsumme joukkoa $A$ \emph{aakostoksi}.

    \begin{enumerate}
        \item Alkiota $u \in  A^n = \underbrace{A \times \cdots \times A}_{\text{n kappaletta}}$ kutsutaan \emph{sanaksi}. Sanan $u$ pituus on $n$ ja se muodostuu aakoston $A$ alkioista.
        \item Osajoukkoa $C \subseteq A^n$ kutsutaan \emph{koodiksi}.
        \item Alkiota $u \in C \subseteq A^n$ kutsutaan \emph{koodisanaksi}.
        \item Jos joukko $A$ on kunta, niin $A^n$ on $A$-vektoriavaruus. Jos nyt $C \subseteq A^n$ kutsutaan osajoukkoa $C$ \emph{lineaariseksi koodiksi}. Lisäksi jos $\dim_A C = k$ kutsutaan osajoukkoa $C$ $(n, k)$ \emph{lineaariseksi koodiksi}. Jos $A = \Zset_2$ niin osajoukkoa $C$ kutsutaan \emph{lineaariseksi binääri koodiksi}.
    \end{enumerate}
\end{maaritelma}


\begin{lause}\label{thm: kasautumispisteen ympäristössä}
    A on B
\end{lause}

\begin{proof}
    Toimii
\end{proof}

%%%%%%%%%%%%%%%%%%%%%%%%%%%%%%%%%%%%%%%%%%%%%%%%%%%%%%%%%%%%%%%%%%%%
\section{Supremum jonon raja-arvona}%
\label{sec: Supremum jonon raja-arvona}



%%%%%%%%%%%%%%%%%%%%%%%%%%%%%%%%%%%%%%%%%%%%%%%%%%%%%%%%%%%%%%%%%%%%
\begin{thebibliography}{9}
% LaTeX ei oletusarvoisesti lisää lähdeluettelon otsikkoa sisällysluetteloon, joten se on tehtävä tässä erikseen:
\addcontentsline{toc}{chapter}{\bibname}

\bibitem{Mathman}
Mathman, Z. \emph{An elementary course in mathematical analysis}.
New York: Birch and Star, 1990.

\end{thebibliography}


%%%%%%%%%%%%%%%%%%%%%%%%%%%%%%%%%%%%%%%%%%%%%%%%%%%%%%%%%%%%%%%%%%%%
% Jos liitteitä on vain yksi, sitä ei tarvitse numeroida:
\chapter*{Liite}
% Lisätään numeroimaton liite sisällysluetteloon:
\addcontentsline{toc}{chapter}{Liite}

Mahdolliset liitteet sijoitetaan tutkielman loppuun.

% Jos liite on tehty jollain toisella ohjelmalla, niin sen voi liittää tutkielmaan pdf-muodossa makropaketin pdfpages komennolla \includepdf. Poista tällöin edelliset komennot \chapter*{Liite} ja \addcontentsline{toc}{chapter}{Liite} ja lisää liitteen alkuun otsikko ja tarvittaessa liitteen kuvaus esimerkiksi samalla ohjelmalla, jolla liitekin on tehty.
%\includepdf[pages={1-4}]{tiedosto.pdf}


\end{document}
%%%%%%%%%%%%%%%%%%%%%%%%%%%%%%%%%%%%%%%%%%%%%%%%%%%%%%%%%%%%%%%%%%%%
